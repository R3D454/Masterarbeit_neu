


\DeclareGraphicsRule{.pdftex}{pdf}{*}{}


\begin{titlepage}
\singlespacing
\definecolor{hsu-rot}{rgb}{0.6745,0.0118,0.2275}
\definecolor{hsu-grau}{rgb}{0.5608,0.5569,0.549}
\hfill\includegraphics[width=5cm]{hsu_4c} 	
\vspace{5cm}\\
\color{hsu-rot}



\large\bfseries
Henry Winkel\\[2ex]
\color{hsu-grau}
Entwurf einer Klassenhierarchie für militärische Simulationen in DIS \\%%\\[2ex]
Developing a class hierarchy for military simulations in DIS
\vfill
Masterarbeit
\normalsize
\vspace{1cm}

\begin{center}
\mbox{\rule{-0.13\paperwidth}{0ex}\color{hsu-rot}{\rule{1.05\paperwidth}{2mm}} }

\end{center}
\vspace{1.7cm}
\color{hsu-grau}
Fakultät für Elektrotechnik	
\begin{tabbing}
Weiterer Prüfer: \= \kill
Studiengang:\>
Informatik-Ingenieurwesen / ET2014\\
Matr.-Nr.:\> 874650  \\
Übernahme:\>30. Mai 2018\\
Betreuer:\>Univ.-Prof.~Dr.~phil.~nat.~habil.~Bernd Klauer
\\Weiterer Prüfer:\>Univ.-Prof.~Dr.-Ing.~habil.~Udo Zölzer 	
	
\end{tabbing}
\vspace{-2.5cm}
\end{titlepage}

\pagenumbering{roman}
\cleardoublepage
\section*{Erklärung}
% Text nicht ändern, er muss der SPO entsprechen
Hiermit versichere ich, dass ich diese Arbeit selbständig verfasst,
keine anderen als die im Quellen- und Literaturverzeichnis genannten Quellen
und Hilfsmittel, insbesondere keine dort nicht genannten Internet-Quellen 
benutzt,
alle aus Quellen und Literatur wörtlich oder sinngemäß entnommenen Stellen als
solche kenntlich gemacht habe und dass die auf einem elektronischen
Speichermedium abgegebene Fassung der Arbeit der gedruckten entspricht. 

\vspace{1cm}
Hamburg, \today\\[-1ex]
\mbox{}\hspace{2.2cm}\parbox[t]{4cm}{\centering \dotfill\\(Datum)}\hspace{0.5cm}
					\parbox[t]{8cm}{\centering \dotfill\\(Unterschrift)}
					
\cleardoublepage
%\chapter*{Vorwort}
%Dank an diverse Personen etc. (nur wenn unbedingt nötig; ist sonst
%für studentische Arbeit nicht üblich)
%\listoffigures
\listoftables
\lstlistoflistings
%\newpage
\addcontentsline{toc}{chapter}{Listings}

\newpage
\chapter*{Abkürzungsverzeichnis}
\addcontentsline{toc}{chapter}{Abkürzungsverzeichnis} 
% Sortierung alphabetisch, nach Reihenfolge des Auftretens oder anderer
% sinnvoller Reihenfolge
%
\begin{acronym}[XXXXXX]
	\setlength{\itemsep}{-\parsep}
	\acro{dis}[DIS]{Distributed Interactive Simulation}
	\acro{ieee}[IEEE]{Institute of Electrical and Electronics Engineers}
	\acro{pdu}[PDU]{Protocol Data Units}
	\acro{espdu}[ESPDU]{Entity State PDU}
	\acro{oop}[OOP]{objektorientierte Programmierung }
	\acro{uml}[UML]{Unified Modeling Language}
	\acro{ide}[IDE]{Integrated Development Environment}
	\acro{hla}[HLA]{High-Level Architecture}
	\acro{vls}[VLS]{Vertical Launching System }
	\acro{udp}[UDP]{User Datagram Protocol}
	
\end{acronym}
\newpage

%% General technical informatik stuff
\newacronym{ac97}{AC '97}{Audio Codec '97}
\newacronym{cpu}{CPU}{Central Processing Unit}
\newacronym{intel}{INTEL}{Intel Cooperation}
\newacronym{prhs}{PRHS}{Partial Reconfigurable Heterogeneous System}
\newacronym{spif}{S/PDIF}{Sony/Philips Digital Interface} 







\cleardoublepage
\tableofcontents


%\cleardoublepage
%\chapter*{Häufige verwendete Formelzeichen und Abkürzungen}
% Sortierung alphabetisch, nach Reihenfolge des Auftretens oder anderer
% sinnvoller Reihenfolge
%

%\printglossary[type=\acronymtype,style=long]
\newpage

%\begin{tabbing}
%Längstes Formelsymbol \= \kill \\

%\end{tabbing}
